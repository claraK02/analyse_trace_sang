\documentclass[a4paper]{article}
\usepackage[french]{babel}
\usepackage[utf8]{inputenc}
\usepackage[T1]{fontenc}
\usepackage{booktabs}   % For better-looking horizontal lines
\usepackage{array}      % For adjusting row spacing
\usepackage{csquotes}   % pour éviter un warning
\usepackage{graphicx}   % pour les images
\usepackage{hyperref}   % pour les références
\usepackage{float}      % pour les figures
\usepackage{amsmath}    % pour les formules mathématiques
% \usepackage{multicol}   % pour les 2 colonnes
\usepackage{subcaption} % pour les sous-figures
% \usepackage{caption}    % pour les légendes
% \usepackage{fullpage}   % mets les marges en petits 
% \usepackage{placeins}   % For     \FloatBarrier command
\usepackage{appendix}  % For appendices

\usepackage[style=ieee,backend=biber]{biblatex}
\addbibresource{ref.bib}

\hypersetup{
    colorlinks=true,
    linkcolor=blue,   % Couleur des liens internes (table des matières, références)
    citecolor=green,  % Couleur des liens vers les références bibliographiques
    filecolor=magenta,% Couleur des liens vers les fichiers
    urlcolor=blue     % Couleur des liens vers les URL
}

\title{Rapport Projet 3A \\ Analyse de traces de sang}
% \subtitle{Analyse de trace de sang}
\author{Cléa Han, Yanis Labeyrie et Adrien Zabban}
% \institute{Ecole Centrale Méditerranée, 13013 Marseille, France}
\date{Mars 2024}

\begin{document}

\maketitle
% \bigskip
% \tableofcontents
% \newpage

\section{Introduction}

Notre projet 3A s'intéresse à l'analyse de traces de sang dans le cadre du travail de l'expert criminalistique Philippe Esperança. En effet, l'objectif est de réaliser une intelligence artificielle pour assister et faciliter le travail d'analyse de scènes de crimes présentant du sang. 

Les travaux de Philippe Esperança sur l'analyse des traces de sangs~\cite{PhilippeEsperança}, a abouti à la classification de ces traces en 19 classes distinctes qui sont listées dans la Table~\ref{tab:classes}. Notre but est alors de faire un modèle de machine learning capable de prédire la classe pour une image de trace de sang.

\begin{table}[ht]
    \centering
    \begin{tabular}{|ll|}
        \hline
        \textbf{Classe des types de trace de sang} &  \\
        \hline
        1- Modèles Traces passives & 2- Modèles Goutte à Goutte \\
        3- Modèle Transfert par contact & 4- Modèle Transfert glissé \\
        5- Modèle Altération par contact & 6- Modèle Altération glissée \\
        7- Modèle d'Accumulation & 8- Modèle Coulée \\
        9- Modèle Chute de volume & 10- Modèle sang Propulsé \\
        11- Modèle d'éjection & 12- Modèle Volume Impacté \\
        13- Modèle Imprégnation & 14- Modèle Zone d'interruption \\
        15- Modèle d'impact & 16- Modèle Foyer de modèle d'impact \\
        17- Modèle Trace gravitationnelle & 18- Modèle Sang expiré \\
        19- Modèle Trace d'insecte & \\
        \hline
    \end{tabular}
    \caption{Liste des 19 modèles de trace de sang}
    \label{tab:classes}
\end{table}

\section{Données}

Philippe Esperança nous a fourni deux bases de données. La première contient des images de traces de sang reproduites en laboratoire. La deuxième correspond à des images issues de scène de crime.
Cependant, il y avait la présence d'une classe trop minoritaire parmi ces 19 classes, qui est la classe de trace d'insectes possédant uniquement quatre images. Cette classe a donc été retirée afin de garder une certaine distribution relativement équilibrée. Nous avons donc travaillé avec 18 classes.

\subsection{Données de laboratoire}

Dans un premier temps, nous avons pu manipuler des données de laboratoire, c'est-à-dire des images de traces de sang reproduites en laboratoire sur des fonds réguliers, hors des scènes de crimes. Ces fonds sont de quatre types différents : bois, linoleum (lino), carrelage et papier. Ces données sont composées de $10978$ images.
La Figure~\ref{fig: labs images} présente des images de taches de sang reproduites en laboratoire. La Table~\ref{tab: images of all classes} en annexe montre un exemple de trace de sang pour chacune des 18 classes.

\begin{figure}[ht]
    \centering
    \begin{subfigure}{0.40\linewidth}
        \centering
        \includegraphics[width=\linewidth]{../asset/data_labo/4_papier_1586.jpg}
        \caption{Modèle Transfert glissé sur un fond de papier}
    \end{subfigure}
    \begin{subfigure}{0.40\linewidth}
        \centering
        \includegraphics[width=\linewidth]{../asset/data_labo/8_coulée_4526.jpg}
        \caption{Modèle de Coulée sur fond de lino}
    \end{subfigure}
    \caption{Deux images de laboratoire}
    \label{fig: labs images}
\end{figure}

Nous avons réparti ces données de laboratoire en 80\% dans nos données d'entraînement, 10\% dans nos données de validation et 10\% dans nos données de test. La Figure~\ref{fig:distribution labo} nous montre la distribution des données de laboratoire sur chacune des classes et chacun des datasets.

\begin{figure}[ht]
    \centering
    \includegraphics[width=0.8\linewidth]{../asset/distribution_train_val_test.png}
    \caption{Distribution des données de laboratoire sur les 18 classes selon le dataset d'entraînement, de validation et de test.}
    \label{fig:distribution labo}
\end{figure}

\subsection{Données réelles issues de scènes de crime}
Après l'élaboration de nos éventuels modèles pour la problématique traitée, nous avons pu manipuler des données dites réelles. Ce sont des données prises directement sur les scènes de crimes, qui sont composées de 245 images. Les images sont alors relativement moins consistantes et plus hétérogènes que les données de laboratoire. En effet, ces images sont donc issues de prises réalisées le plus souvent par la police scientifique, qui ne prend pas en compte les conditions consistantes de prise de photo respectées dans les données de laboratoire prises par Philippe Esperança. La Figure~\ref{fig: reals images} montre des exemples d'images réelles. On peut dès maintenant s'apercevoir que ces données vont être beaucoup plus compliquées à analyser pour nos modèles de deep learning au vu des nombreux objets présents sur les photos.

\begin{figure}[ht]
    \centering
    \begin{subfigure}{0.40\linewidth}
        \centering
        \includegraphics[width=\linewidth]{../asset/data_real/12.jpg}
        \caption{Modèle Volume Impacté}
    \end{subfigure}
    \begin{subfigure}{0.40\linewidth}
        \centering
        \includegraphics[width=\linewidth]{../asset/data_real/15.jpg}
        \caption{Modèle d'impact}
    \end{subfigure}
    \caption{Deux images de scènes de crime}
    \label{fig: reals images}
\end{figure}

Nous avons réparti ces données réelles à 60\% dans nos données d'entraînement, à 10\% dans nos données de validation et à 30\% dans nos données de test. En effet, une plus grande proportion d'images a été attribuée au test des données réelles afin d'avoir un test relativement plus représentatif. Cet ensemble de données de test est composé de 73 images. La Figure~\ref{fig:distribution real} nous montre la distribution des données réelles sur chacune des classes et chacun des datasets.

\begin{figure}[ht]
    \centering
    \includegraphics[width=0.8\linewidth]{../asset/distribution_train_val_test.png}
    \caption{Distribution des données de scène de crimes selon le dataset d'entraînement, de validation et de test.}
    \label{fig:distribution real}
\end{figure}

\subsection{Data Processing}
Les images que l'on a reçues sont en couleurs et ont une taille variable allant de $2000\times2000$ à $10000\times10000$ pixels. Nous avons donc redimensionné toutes nos images en $256\times256$. Pour l'entraînement de nos modèles deep learning, nous avons fait des symétries horizontales et verticales. Nous n'avons pas fait de rotation sur les images, car Philippe Esperança nous a expliqué qu'il ne prenait que des photos en face des taches de sang (donc sans rotation). Donc il n'est pas nécessaire d'apprendre les rotations des images. Nous avons aussi joué sur le contraste et la luminescence des images. Pour la validation, le test et l'inférence, nous n'avons pas mis de data augmentation.

\section{Modèles}

Pour aborder l'analyse de traces de sang, nous avons décidé d'utiliser le modèle Resnet 50~\cite{ResNet} pré-entraîné sur ImageNet selon certains poids indiqués dans la bibliographie~\cite{torchvision}, selon diverses approches. Nous avons utilisé la crossentropy pour notre fonction de coût, et Adam~\cite{adam} pour l'optimizer.

\subsection{ResNet linear probe}
Nous avons dans un premier temps retiré la dernière couche dense du modèle, qui était destinée à classifier sur 1000 catégories, puis effectuer du linear probing. Nous avons donc gelé tous les poids du Resnet et nous avons remplacé la dernière couche dense par 2 couches denses (qui elles sont apprenables) pour avoir une dernière couche dense à 18 neurones\footnote{la sortie correspond alors aux 18 modèles de taches de sang.}. La couche dense intermédiaire est de taille 64, et nous avons fixé un dropout à $0.1$. La Figure~\ref{fig:resnet} représente ce modèle linear probe ResNet. Les modèles utilisant le linear probing seront désignés par LP Resnet dans la suite.

\begin{figure}[ht]
    \centering
    \includegraphics[width=0.7\linewidth]{../asset/Resnet.png}
    \caption{Schéma du modèle Resnet}
    \label{fig:resnet}
\end{figure}

\subsection{Réentraînement de tout le ResNet}
Nous avons également testé la méthode LP ResNet sans geler les poids de convolution du ResNet. Les modèles concernés par cette méthode seront désignés par "AWL ResNet" pour All weight learnable.

\subsection{Modèle Adversarial}

Un des points important pour la classification d'image est de pouvoir détecter la tache de sang et de la détacher du fond (background). Nous avons alors implémenté un entraînement adversarial. En plus du modèle ResNet, nous avons rajouté un modèle de MLP (Multilayer Perceptron) composé de deux couches fully connected layer qui prend en entrée la sortie de l'avant-dernière couche dense de notre Linear Probe ResNet, et prédit le background. Le but est d'alors de faire en sorte que le modèle ResNet ne possède pas d'informations sur le background de l'image dans son espace latent. La Figure~\ref{fig:resnet_adv} montre ces deux modèles. Nous appellerons cette approche "modèle Adversarial".

\begin{figure}[ht]
    \centering
    \includegraphics[width=0.7\linewidth]{../asset/Resnet_adv.png}
    \caption{Schéma du modèle Resnet adversarial}
    \label{fig:resnet_adv}
\end{figure}

Nous avons fait un entraînement adversarial pour entraîner les deux modèles simultanément. Nous avons utilisé la fonction de coût définie dans la Formule~\ref{eq: advloss}, où $CE_{tache}$ est la crossentropy entre la prédiction du modèle ResNet et la classe de la tache de sang, et $CE_{background}$ est la crossentropy de la prédiction du modèle adversarial et le background de l'image. Le paramètre $\alpha$ est un hyperparamètre que l'on va optimiser dans la section~\ref{sec: random search}.
\begin{equation}
    L_{adv} = \frac{CE_{tache}}{\alpha CE_{background}}
    \label{eq: advloss}
\end{equation}

La loss $L_{adv}$ est rétro-propagée dans le modèle ResNet tandis que la loss $CE_{background}$ est alors propagée dans le modèle adversarial.

\subsection{Fine-tune des modèles sur les données réelles}
Une fois que nous avons appris nos modèles LP ResNet et AWL ResNet sur les données de laboratoires, nous avons fine-tune ces modèles sur les données réelles (données issues de scène de crime). Nous avons appelé ces modèles respectivement FT LP Resnet et FT AWL ResNet.

Nous n'avons pas pu fine-tune le modèle Adversarial, car les taches de sang ne sont pas sur les fonds unis, et par conséquent, il n'est pas possible de fine-tune le modèle prédicteur de background sur ces données.

\section{Métriques}
Pour comparer les performances de nos modèles, nous avons utilisé les métriques suivantes : l'accuracy micro (indique le pourcentage de réussite d'un modèle), l'accuracy macro (indique la moyenne du pourcentage de réussite de chacune des classes), la précision, le rappel, le f1-score, le top~3 (indique le pourcentage de trouver la réponse parmi les 3 classes les plus prédites par le modèle). Toutes ces métriques sont des valeurs qui vont de 0 à 100, où 100 est le meilleur score. 
Les procédures de calcul de ces métriques sont explicitées en détail dans l'annexe~\ref{sec: metrics}.

\section{Apprentissage des modèles}

\subsection{Trouver le meilleur learning rate}
Afin de maximiser nos potentielles performances, nous avons commencé à effectuer un Grid-Search pour trouver le meilleur learning rate possible. Nous avons lancé l'entraînement du LP ResNet sur 5 epochs avec les learning rates suivants : $0.01, 0.005, 0.001, 0.0005, 0.0001$. La Table~\ref{tab:gridsearch} montre alors les résultats de ce Grid-Search. Pour la suite, nous avons pris un learning rate à $0.0005$.

\begin{table}[ht]
    \centering
    \begin{tabular}{cccccc}
    \toprule
    learning rate & acc micro & acc macro & f1-score & top 3 \\
    \midrule
    0.01 & 85.1 & 78.8 & 78.1 & 49.4 \\
    0.005 & 89.1 & 84.7 & 83.6 & \textbf{52.5} \\
    0.001 & 90.4 & 85.7 & 84.8 & 50.5 \\
    0.0005 & \textbf{91.1} & \textbf{86.8} & \textbf{86.0} & 49.1 \\
    0.0001 & 84.9 & 78.6 & 77.4 & 50.4 \\
    \bottomrule
    \end{tabular}
    \caption{Résultat de validation à la fin des entraînements des modèles LP ResNet avec différents learning rate.}
    \label{tab:gridsearch}
\end{table}

\subsection{Entraînement du modèle LP ResNet}
Nous avons entraîné le modèle LP ResNet sur les données de laboratoire sur 10 epochs et un learning rate de $0.0005$. La Figure~\ref{fig: train LP ResNet} montre les courbes d'apprentissage de ce modèle.

\begin{figure}[ht]
    \centering
    \begin{subfigure}{0.32\textwidth}
        \centering
        \includegraphics[width=\linewidth]{../logs/resnet_img256_0/crossentropy.png}
        \caption{crossentropy}
    \end{subfigure}
    \begin{subfigure}{0.32\textwidth}
        \centering
        \includegraphics[width=\linewidth]{../logs/resnet_img256_0/acc micro.png}
        \caption{acc micro}
    \end{subfigure}
    \begin{subfigure}{0.32\textwidth}
        \centering
        \includegraphics[width=\linewidth]{../logs/resnet_img256_0/acc macro.png}
        \caption{acc macro}
    \end{subfigure}
    \caption{Valeurs de la loss et des accuracy d'entraînement (en bleue) et de validation (en orange) en fonction des epochs durant l'entraînement du modèle LP ResNet.}
    \label{fig: train LP ResNet}
\end{figure}

\subsection{Entraînement du modèle AWL ResNet}
Nous avons entraîné le modèle LP ResNet sur les données de laboratoire sur 20 epochs. Ici, nous avons mis un learning rate de $0.0001$ pour qu'il apprenne moins vite et ne change pas trop ses paramètres dans les couches de convolutions. Nous n'avons pas pu faire une méthode de Grid Search comme le modèle de LP ResNet, en raison du temps d'apprentissage qui est plus long. La Figure~\ref{fig: train AWL ResNet} montre les courbes d'apprentissage de ce modèle.

\begin{figure}[ht]
    \centering
    \begin{subfigure}{0.32\textwidth}
        \centering
        \includegraphics[width=\linewidth]{../logs/resnet_allw_img256_2/crossentropy.png}
        \caption{Crossentropy}
    \end{subfigure}
    \begin{subfigure}{0.32\textwidth}
        \centering
        \includegraphics[width=\linewidth]{../logs/resnet_allw_img256_2/acc micro.png}
        \caption{acc micro}
    \end{subfigure}
    \begin{subfigure}{0.32\textwidth}
        \centering
        \includegraphics[width=\linewidth]{../logs/resnet_allw_img256_2/acc macro.png}
        \caption{acc macro}
    \end{subfigure}
    \caption{Valeurs de la loss et des accuracy d'entraînement (en bleue) et de validation (en orange) en fonction des epochs durant l'entraînement du modèle AWL ResNet.}
    \label{fig: train AWL ResNet}
\end{figure}

\subsection{Trouver les hyperparamètres pour le modèle Adversarial}
\label{sec: random search}
On cherche ici à trouver le meilleur learning rate pour optimiser le ResNet $lr_{res}$ et pour celui adversarial $lr_{adv}$. On cherche aussi le meilleur paramètre $\alpha$, utilisé dans la formule~\ref{eq: advloss}. On a alors implémenté un Random Search~\cite{randomsearch} et fait 10 expériences de 5 epochs, en prenant à chaque fois des paramètres dans la Table~\ref{tab:random search possibilities}\footnote{On a fait un Random Search et pas un Grid-Search, car il n'est pas possible de tester toutes les combinaisons possibles, qui sont au nombre de 200.}

\begin{table}[ht]
    \centering
    \begin{tabular}{cc}
        \toprule
        Paramètres & Valeurs possibles\\
        \midrule
        $lr_{res}$ & 0.01, 0.005, 0.001, 0.005, 0.0001\\
        $lr_{adv}$ & 0.01, 0.005, 0.001, 0.005, 0.0001\\
        $\alpha$ & 0.001, 0.1, 0.5, 1, 2, 5, 10, 100\\
        \bottomrule
    \end{tabular}
    \caption{Liste des valeurs possibles pour les hyperparamètres testés dans le Random Search}
    \label{tab:random search possibilities}
\end{table}

La Table~\ref{tab:random search results} montre les résultats de ce Random Search. \textit{res acc micro} représente l'accuracy micro de la partie ResNet (de même pour \textit{res acc macro}), et \textit{adv acc micro} représente l'accuracy micro de la partie Adversarial. Nous avons donc choisi de prendre les hyperparamètres de l'expérience numéro 2.

\begin{table}[ht]
    \centering
    \begin{tabular}{ccccccc}
    \toprule
    numéro & res acc micro & res acc macro & adv acc micro & $lr_{res}$ & $lr_{adv}$ & $\alpha$ \\
    \midrule
    0 & 79.3 & 72.8 & 82.5 & 0.1 & 1 & 10 \\
    1 & 89.6 & 84.7 & 16.1 & 0.1 & 1 & 0.1 \\
    2 & \textbf{90.8} & \textbf{86.2} & 20.2 & 0.5 & 0.01 & 0.1 \\
    3 & 85.3 & 81.2 & 44.6 & 0.1 & 1 & 0.5 \\
    4 & 88 & 85.7 & 72 & 0.5 & 0.5 & 2 \\
    5 & 89.4 & 85.4 & 56.8 & 0.01 & 0.1 & 0.5 \\
    6 & 89.5 & 85.4 & 71 & 0.1 & 0.1 & 1 \\
    7 & 87.2 & 82.2 & 71.7 & 1 & 0.01 & 1 \\
    8 & 85.3 & 80.3 & \textbf{85.3} & 0.1 & 0.5 & 10 \\
    9 & 86.7 & 83.2 & 84.8 & 0.1 & 0.01 & 10 \\
    \bottomrule
    \end{tabular}
    \caption{Résultat de validation à la fin des entraînements des modèles Adversarial avec différents hypermaramètres.}
    \label{tab:random search results}
\end{table}

\subsection{Entraînement du modèle Adversarial}
Nous avons donc entraîné le modèle Adversarial sur les données de laboratoire sur 20 epochs avec les hyperparamètres trouvés dans la section~\ref{sec: random search}. La Figure~\ref{fig: train adv} montre les courbes d'apprentissages de ce modèle.

\begin{figure}[ht]
    \centering
    \begin{subfigure}{0.32\textwidth}
        \centering
        \includegraphics[width=\linewidth]{../logs/adv_img256_1/crossentropy.png}
        \caption{Loss $L_{adv}$}
    \end{subfigure}
    \begin{subfigure}{0.32\textwidth}
        \centering
        \includegraphics[width=\linewidth]{../logs/adv_img256_1/resnet_acc micro.png}
        \caption{res acc micro}
    \end{subfigure}
    \begin{subfigure}{0.32\textwidth}
        \centering
        \includegraphics[width=\linewidth]{../logs/adv_img256_1/adv_acc micro.png}
        \caption{adv acc micro}
    \end{subfigure}
    \caption{Valeurs de la loss et des accuracy d'entraînement (en bleue) et de validation (en orange) en fonction des epochs durant l'entraînement du modèle Adversarial.}
    \label{fig: train adv}
\end{figure}

\subsection{Fine-tune des modèles LB ResNet et AWL ResNet}
Nous avons ensuite appris les modèles FT LB ResNet et FT AWL ResNet sur 10 epochs qui sont des modèles fine-tune sur les données réelles à partir des modèles respectivement LB ResNet et AWL ResNet sur 10 epochs. La Figure~\ref{fig: finetune} montre les courbes d'apprentissages de ce modèle.

\begin{figure}[H]
    \centering
    \begin{subfigure}{0.35\textwidth}
        \centering
        \includegraphics[width=\linewidth]{../logs/retrain_resnet_img256_0/crossentropy.png}
        \caption{crossentropy de FT LP ResNet}
    \end{subfigure}
    \begin{subfigure}{0.35\textwidth}
        \centering
        \includegraphics[width=\linewidth]{../logs/retrain_resnet_allw_img256_2/crossentropy.png}
        \caption{crossentropy de FT AWL ResNet}
    \end{subfigure}
    \caption{Valeurs de la loss d'entraînement (en bleue) et de validation (en orange) en fonction des epochs durant l'entraînement des modèles fine-tune sur les données réelles.}
    \label{fig: finetune}
\end{figure}

\section{Interprétabilité}


Afin de répondre à l’aspect "boîte noire" du modèle utilisé, nous avons im-
plémenté de l’interprétabilité dans notre modèle à l’aide de Grad-CAM [6].
Cette méthode permet de fournir une explication visuelle vis-à-vis des déci-
sions de classification issue de notre modèle, permettant ainsi de rajouter une
certaine légitimité relative face à nos analyses de traces de sang faisant partie
d’un processus judiciaire. 

L'algorithme Grad-CAM (Gradient-weighted Class Activation Mapping) est une technique utilisée pour rendre les réseaux de neurones convolutifs (CNN) plus interprétables dans le domaine de la classification.
Il fonctionne en générant des cartes de chaleur (heatmaps) qui mettent en évidence les zones importantes d'une image qui contribuent le plus à la prédiction d'une classe spécifique.
Pour ce faire, Grad-CAM calcule les dérivées des scores de sortie de la classe cible par rapport aux caractéristiques de la dernière couche convolutive du CNN.
Ces dérivées sont ensuite globalement moyennées pour obtenir les poids d'importance de chaque carte de caractéristiques.
Enfin, les cartes de caractéristiques sont pondérées par les poids d'importance et combinées pour obtenir la carte de chaleur Grad-CAM, qui peut être superposée à l'image d'origine pour une visualisation plus intuitive.
Cette approche permet de mieux comprendre les décisions prises par le CNN et d'améliorer la confiance dans les prédictions du modèle.

Par exemple, nous avons remarqué que parfois certaines des réglettes permettant de donner l'échelle de la photo étaient détectées comme des traces de sang ou induisaient un biais dans la classification.
En utilisant Grad-CAM, nous avons pu identifier que le modèle se focalisait sur ces réglettes pour prendre sa décision, ce qui nous a permis de mieux comprendre
les erreurs de classification et d'améliorer la qualité des prédictions.
[figure montrant une tache avec une réglette.]
Un autre avantage de Grad-CAM est qu'il permet de vérifier si le modèle se focalise sur la bonne tâche de sang
dans le cas où plusieurs tâches sont présentes sur une même image.

Dans l'exemple suivant, nous voyons que c'est la tâche de gauche qui est détectée par le modèle, ce qui est confirmé par la carte de chaleur Grad-CAM qui met en évidence les zones importantes de l'image pour cette tâche.
[figure montrant 2 images côte à côte avec les cartes de chaleur Grad-CAM correspondantes].

\section{Résultats}

Les résultats de test sur les données de laboratoire~\ref{tab:results_lab} indiquent de meilleures performances pour le modèle Resnet qui a été réentraîné sur tous ses poids, selon l'accuracy et le F1-score. 

\begin{table}[ht]
  \centering
    \begin{tabular}{cccccc}
    \toprule
    Modèles & Acc Micro & Acc Macro & F1-score & Top 3 \\
    \midrule
    LP ResNet & 95.2 & 94.3 & 94.7 & \textbf{99.8} \\
    AWL ResNet & \textbf{97.3} & \textbf{97.1} & \textbf{96.2} & \textbf{99.8} \\
    Adversarial & 93.4 & 91.8 & 91.8 & \textbf{99.8} \\
    \bottomrule
    \end{tabular}
    \caption{Résultats de test sur les données de laboratoire}
    \label{tab:results_lab}
\end{table}

Les résultats de test sur les données de laboratoire~\ref{tab: results_real} indiquent de meilleures performances pour le modèle Resnet qui était entraîné entièrement sur les données de laboratoire, puis fine-tuné sur les données réelles, selon l'accuracy et le F1-score. 

\begin{table}[ht]
  \centering
    \begin{tabular}{ccccccc}
    \toprule
    Modèles & Acc Micro & Acc Macro & F1-score & Top 3 \\
    \midrule
    LP ResNet & 12.9 & 6.0 & 4.0 & 19.0 \\
    FT LP ResNet & 11.8 & 6.1 & 6.4 & 26.0 \\
    AWL Resnet & 17.2 & 13.8 & 8.1 & 28.7 \\
    FT AWL ResNet & \textbf{41.9} & \textbf{33.4} & \textbf{26.9} & \textbf{51.7} \\
    Adversarial & 11.8 & 5.7 & 3.7 & 16.7 \\
    \bottomrule
    \end{tabular}
    \caption{Résultats de test sur les données réelles}
    \label{tab: results_real}
\end{table}

\section{Conclusion}

Notre projet s'inscrit dans une volonté d'optimiser le travail d'analyse de traces de sang en criminologie, notamment pour l'expert international Philippe Esperança. 
Nous avons aussi testé d'autres méthode que nous avons détaillées dans la section~\ref{sec: pistes infructruces}.

En effet, notre encadrant a pu recevoir notre projet sous la forme d'une interface qui fonctionne localement sur son appareil de travail. En effet, ce modèle d'analyse de traces de sang ne doit pas avoir accès à internet afin d'éviter tout risque d'attaque ou de fuite de données. Pouvoir faire tourner le projet en local pour l'expert criminalistique lui permet de nous faire confiance vis-à-vis de l'outil implémenté. Il a pu apprécier et être satisfait de notre rendu.

Notre projet fait également l'objet d'une documentation afin qu'il puisse être repris aisément à l'avenir. 

\printbibliography

% \newpage
\begin{appendices}

\section{Images de laboratoire}
\label{sec: annexe}

\begin{table}[H]
    \centering
    \begin{tabular}{ccc}
        % \hline
        \includegraphics[width=0.20\linewidth]{../asset/data_labo/1_bois_350.jpg} & \includegraphics[width=0.20\linewidth]{../asset/data_labo/2_carrelage_523.jpg}& \includegraphics[width=0.20\linewidth]{../asset/data_labo/3_lino_888.jpg} \\
        1- Traces passives & 2- Goutte à Goutte & 3- Transfert par contact \\
        % \hline
        \includegraphics[width=0.20\linewidth]{../asset/data_labo/4_papier_1586.jpg} & \includegraphics[width=0.20\linewidth]{../asset/data_labo/5_carrelage_5605.jpg} & \includegraphics[width=0.20\linewidth]{../asset/data_labo/6_bois_604.jpg} \\
        4- Transfert glissé & 5- Altération par contact & 6- Altération glissée \\
        % \hline
        \includegraphics[width=0.20\linewidth]{../asset/data_labo/7_carrelage_5507.jpg} & \includegraphics[width=0.20\linewidth]{../asset/data_labo/8_coulée_4526.jpg} & \includegraphics[width=0.20\linewidth]{../asset/data_labo/9_papier_6375.jpg} \\
        7- Accumulation & 8- Coulée & 9- Chute de volume \\
        % \hline
        \includegraphics[width=0.20\linewidth]{../asset/data_labo/10_lino_933.jpg} & \includegraphics[width=0.20\linewidth]{../asset/data_labo/11_carrelage_905.jpg} & \includegraphics[width=0.20\linewidth]{../asset/data_labo/12_bois_326.jpg} \\
        10- Sang Propulsé & 11- éjection & 12- Volume impacté \\
        % \hline
        \includegraphics[width=0.20\linewidth]{../asset/data_labo/13_carrelage_7374.jpg} & \includegraphics[width=0.20\linewidth]{../asset/data_labo/14_lino_6320.jpg} & \includegraphics[width=0.20\linewidth]{../asset/data_labo/15_p1040.jpg} \\
        13- Imprégnation & 14- Zone d'interruption & 15- Modèle d'impact \\
        % \hline
        \includegraphics[width=0.20\linewidth]{../asset/data_labo/16_carrelage_598.jpg} & \includegraphics[width=0.20\linewidth]{../asset/data_labo/17_papier_8323.jpg} & \includegraphics[width=0.20\linewidth]{../asset/data_labo/18_bois_4727.jpg}\\
        16- Foyer de modèle d'impact & 17- Trace gravitationnelle & 18- Sang expiré \\
        % \hline
    \end{tabular}
    \caption{Classe des données de laboratoire et leur exemple en image}
    \label{tab: images of all classes}
\end{table}

\section{Calcul des métriques}
\label{sec: metrics}
\subsection{Accuracy micro}
L'accuracy est le pourcentage de réussite d'un modèle. Elle est calculée par la formule suivante :
\begin{equation}
    \text{Accuracy micro} = \frac{\text{Nombre de prédictions correctes}}{\text{Nombre total de prédictions}}
\end{equation}

\subsection{Accuracy macro}
L'accuracy macro est la moyenne des accuracy de chaque classe. Elle est calculée par la formule suivante :
\begin{equation}
    \text{Accuracy macro} = \frac{1}{N} \sum_{i=1}^{n} \frac{\text{Nombre de prédiction correctes de la classe } i}{\text{Nombre total de prédictions de la classe } i}{}
\end{equation}

\subsection{Précision}
La précision est le pourcentage de prédictions correctes parmi les prédictions positives. Elle est calculée par la formule suivante :
\begin{equation}
    \text{Précision} = \frac{\text{Nombre de vrais positifs}}{\text{Nombre de vrais positifs + Nombre de faux positifs}}
\end{equation}

\subsection{Rappel}
Le rappel est le pourcentage de prédictions correctes parmi les vrais labels positifs. Il est calculé par la formule suivante :
\begin{equation}
    \text{Rappel} = \frac{\text{Nombre de vrais positifs}}{\text{Nombre de vrais positifs + Nombre de faux négatifs}} 
\end{equation}

\subsection{F1-score}
Le F1-score est la moyenne harmonique de la précision et du rappel. Il est calculé par la formule suivante :
\begin{equation}
    \text{F1-score} = 2 \times \frac{\text{Précision} \times \text{Rappel}}{\text{Précision} + \text{Rappel}}
\end{equation}

\subsection{Top 3}
Le Top 3 est le pourcentage de trouver la réponse parmi les 3 classes les plus prédites par le modèle. Il est calculé par la formule suivante :
\begin{equation}
    \text{Top 3} = \frac{\text{Nombre de prédictions correctes parmi les 3 premières prédictions}}{\text{Nombre total d'exemples}}
\end{equation}

\section{Pistes infructueuses}
\label{sec: pistes infructruces}
Au cours de ce projet, nous avons exploré plusieurs pistes qui se sont avérées infructueuses. Nous les présentons ici pour que le lecteur puisse comprendre les raisons pour lesquelles nous avons choisi de ne pas les poursuivre.

Un des objectifs de ce projet était de parvenir à interpréter les choix des modèles de machine learning que nous avons entrainé.
Pour cela, nous avons tenté de nous inspirer du travail réalisé précédemment par l'équipe d'experts criminalistiques. En effet,
leur approche s'inspire de la classification des espèces animales et végétales par les biologistes. Ils ont cherché à identifier des caractéristiques
déterminantes pour chaque trace de sang et ainsi à classifier les traces de sang grâce à un arbre de décision
portant sur ces caractéristiques (par exemple: la forme, la taille, présence de tâches millimétriques près des tâches centimétriques, etc..).
Nous avons donc tenté de reproduire cette approche en utilisant des techniques de feature ingenering pour extraire des caractéristiques
fondamentales des traces de sang souis formes de critères mathématico-géométriques. Pour cela, nous avons d'abord tenté de réaliser
un algorithme de segmentation non supervisé par détection de contours pour extraire les formes des taches de sang, avec des résultats peu concluants. Nous avons ensuite
tenté de réaliser un algorithme mêlant des techniques de traitement d'images (seuils, etc..), d'apprentissage auto-supervisé (Unet~\cite{UNet}) et de géométrie pour extraire des caractéristiques géométriques telle la taille
des tâches satellitaires par rapport à la tâche centrale, la forme des tâches (critère d'ovoïdité, de circularité, etc..). Ces tentatives ont néanmoins
été infructueuses, car les masques de segmentation étaient trop imprécis du fait de la grande variabilité des images de taches de sang. Nous avons également tenté des méthodes de segmentation
adversarial en tirant parti du fait que nous connaissions pour toutes les images la nature du support (bois, lino, carrelage, etc..) pour tenter de segmenter les taches de sang par une approche faiblement supervisée. Ces tentatives ont par ailleurs été infructueuses, car nous 
avions peu de donnée d'entrainement, ce qui rendait tout apprentissage trop instable de manière faiblement supervisée.

Nous avons aussi tenté d'extraire ces caractéristiques (masques de segmentation, critères géométriques) à l'aide de modèle de deep learning
utilisant des techniques dites de "zero-shot classification" (modèles de type CLIP-ViT~\cite{CLip}, etc..) ou zero-shot segmentation (SegmentAnything~\cite{SegmentAnything}). Ces tentatives ont également été infructueuses, car le manque de contraste
sur certaines images de taches de sang ou leur trop grand éclatement rendait les masques de segmentation trop imprécis.


\end{appendices}

\end{document}
