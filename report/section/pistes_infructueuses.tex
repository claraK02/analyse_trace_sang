Au cours de ce projet, nous avons exploré plusieurs pistes qui se sont avérées infructueuses. Nous les présentons ici pour que le lecteur puisse comprendre les raisons pour lesquelles nous avons choisi de ne pas les poursuivre.

Un des objectifs de ce projet était de parvenir à interpréter les choix des modèles de machine learning que nous avons entrainé.
Pour cela, nous avons tenté de nous inspirer du travail réalisé précédemment par l'équipe d'experts criminalistiques. En effet,
leur approche s'inspire de la classification des espèces animales et végétales par les biologistes. Ils ont cherché à identifier des caractéristiques
déterminantes pour chaque trace de sang et ainsi à classifier les traces de sang grâce à un arbre de décision
portant sur ces caractéristiques (par exemple: la forme, la taille, présence de tâches millimétriques près des tâches centimétriques, etc..).
Nous avons donc tenté de reproduire cette approche en utilisant des techniques de feature ingenering pour extraire des caractéristiques
fondamentales des traces de sang souis formes de critères mathématico-géométriques. Pour cela, nous avons d'abord tenté de réaliser
un algorithme de segmentation non supervisé par détection de contours pour extraire les formes des taches de sang, avec des résultats peu concluants. Nous avons ensuite
tenté de réaliser un algorithme mêlant des techniques de traitement d'images (seuils, etc..), d'apprentissage auto-supervisé (Unet~\cite{UNet}) et de géométrie pour extraire des caractéristiques géométriques telle la taille
des tâches satellitaires par rapport à la tâche centrale, la forme des tâches (critère d'ovoïdité, de circularité, etc..). Ces tentatives ont néanmoins
été infructueuses, car les masques de segmentation étaient trop imprécis du fait de la grande variabilité des images de taches de sang. Nous avons également tenté des méthodes de segmentation
adversarial en tirant parti du fait que nous connaissions pour toutes les images la nature du support (bois, lino, carrelage, etc..) pour tenter de segmenter les taches de sang par une approche faiblement supervisée. Ces tentatives ont par ailleurs été infructueuses, car nous 
avions peu de donnée d'entrainement, ce qui rendait tout apprentissage trop instable de manière faiblement supervisée.

Nous avons aussi tenté d'extraire ces caractéristiques (masques de segmentation, critères géométriques) à l'aide de modèle de deep learning
utilisant des techniques dites de "zero-shot classification" (modèles de type CLIP-ViT~\cite{CLip}, etc..) ou zero-shot segmentation (SegmentAnything~\cite{SegmentAnything}). Ces tentatives ont également été infructueuses, car le manque de contraste
sur certaines images de taches de sang ou leur trop grand éclatement rendait les masques de segmentation trop imprécis.